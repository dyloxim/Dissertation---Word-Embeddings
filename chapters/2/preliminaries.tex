\section{Preliminaries}
We begin by defining terms already encountered in mathematical terms.

\begin{definition}[Orthography]\label{def:orthography}
  An orthography is a finite set of symbols.
\end{definition}

\begin{example}\label{ex:orthographies}
  The sets:
  \begin{align}
    &\mathscr{O}_L=\{c\mid\text{$c$ is a symbol that can be copied and pasted the pdf of this report}\}\label{eq:orth-doc}\\
    &\mathscr{O}_P=\{c\mid\text{$c$ is the inscription of a regular polygon with eight sides or less}\}\label{eq:orth-shape}\\
    &\mathscr{O}_K=\{c\mid\text{$c$ is a typable character marked on \autoref{fig:uk-keyboard}}\}\label{eq:orth-keyboard}
  \end{align}
  are all examples of orthographies.
\end{example}
\vspace{6pt}

\begin{figure}[ht]
 \centering
 \includesvg[width=0.9 \columnwidth]{figures/inkscape/uk-keyboard.svg}
 \caption{A standard uk keyboard layout}
 \label{fig:uk-keyboard}
\end{figure}

Note, the term `orthography' in linguistics refers to the conventions of written language in general and is distinct from the use of `an orthography' we use here.

\begin{definition}[Document]
  A document $d$ is a finite sequence $c_0,c_1,\dots,c_n$ ($c\in\mathscr{O}$), where $\mathscr{O}$ is an orthography.
\end{definition}

\begin{example}[This Report]\label{ex:doc-report}
    The sequence of symbols obtained by copy pasting this entire report into a text editor (ordered by starting with the cursor at the top of the file and moving the cursor with the arrow keys one position to the right at a time -including newlines) is a document using the orthography \ref{eq:orth-doc}.
\end{example}
\vspace{6pt}

\begin{example}[Shapes Document]\label{ex:doc-shapes}
    The sequence of shapes: \emph{$\triangle\triangle\square\triangle\triangle\pentagon$} is a document using the orthograpy \ref{eq:orth-shape}.
\end{example}
\vspace{6pt}

\begin{example}[Plato Documents]\label{ex:doc-plato}
    Any of the works of Plato available at the \href{https://www.gutenberg.org/ebooks/author/93}{Project Gutenberg} website, copied and pasted from their plain text formats and using the same procedure described above, are documents using the orthography \ref{eq:orth-keyboard}.
\end{example}

\begin{definition}[Word-Token]
  A word token $w$ has the same attributes as a document; it is a finite sequence $c_0,c_1,\dots,c_n$ ($c\in\mathscr{O}$), where $\mathscr{O}$ is an orthography. We only refer to sequences of symbols as word tokens when they are the output of a tokenizer, however.
\end{definition}

\begin{definition}[Tokenizer]
   A tokenizer $T$ is a function that maps a document to a sequence of word tokens; $T(d)=w_1,w_2,\dots,w_n$. The output is refered to as the 'tokenized' form of the document. We use the shorthand $W_d$ to refer to the tokenized document $T(d)$ when it is clear which tokenization has been used.\footnote{Here we consider tokenization in an abstract sense essentially saying ``A tokenizer is a function that extracts tokens''. A practical discussion of tokenization and text-preprocessing is found in \autoref{sec:preprocessing}}.
\end{definition}

\begin{example}[This Report Tokenized]\label{ex:doc-report-t}
  The document from \autoref{ex:doc-report} if tokenized by treating everything occuring between two spaces (or between a space and the start or end of a line) yields a sequence of word tokens beginning \emph{``THE'' ``UNIVERSITY'' ``OF''}.
\end{example}
\vspace{6pt}

\begin{example}[Shapes Document Tokenized]\label{ex:doc-shapes-t}
  The document from \autoref{ex:doc-shapes} if tokenized using the function:
  \begin{align*}
    &T(d):\quad \texttt{Group repeated symbols in the sequence $d$ into word-tokens}\\
    &\hphantom{T(d):\ }\texttt{(in the order of occurrence).}
  \end{align*}
  produces the output: \emph{``$\triangle$ $\triangle$'', ``$\square$'', ``$\triangle$ $\triangle$'', ``$\pentagon$''}
\end{example}
\vspace{6pt}

\begin{example}[Simple Text Tokenizer]\label{ex:t-simple}
  Given a document $d$ using the orthography $\mathscr{O}_K$ from \autoref{ex:orthographies} apply the following algorithm:
  \begin{algorithm}
    \caption{Simple Text Tokenizer}
    \SetKwData{Tokens}{tokens}
    \SetKwData{NextToken}{next\_token}
    \SetKwData{Doc}{document}
    \SetKwData{Char}{character}
    \SetKwFunction{Append}{append}
    \SetKwFunction{List}{list}
    \SetKwFunction{ToLower}{convert\_to\_lowercase}
    \SetKwInOut{Input}{input}\SetKwInOut{Output}{output}

    \Input{A list of characters: \Doc}
    \Output{A list of word tokens: \Tokens}
    \BlankLine

    \Tokens $\leftarrow$ \List{}\;
    \NextToken $\leftarrow$ \List{}\;
    \For{\Char in \Doc}{
      \uIf{\Char is alphanumeric}{
        \NextToken$\leftarrow$ \Append{\NextToken, \Char}\;
      }
      \ElseIf{character is a space or a newline}{
        \If{\NextToken is not empty}{
          \NextToken$\leftarrow$\ToLower{\NextToken}\;
          \Tokens$\leftarrow$ \Append{\Tokens, \NextToken}\;
          \NextToken$\leftarrow$ \List{}\;
        }
      }
    }
  \end{algorithm}
\end{example}

\begin{definition}[Corpus]
  A corpus $C$ is a finite set of documents $\{d_1, d_2,\dots,d_n\}$.
\end{definition}

\begin{example}[Plato Corpus]\label{ex:corp-plato}
  All the of Plato works referred to in \autoref{ex:doc-plato}, if converted to documents using the same procedure described there, considered together form a corpus of documents. We will refer to this corpus as $C_p$.
\end{example}

\begin{definition}[Vocabulary]
  The vocabulary of a tokenized document $W_d$ is the set of unique word tokens in the tokenization of that document; $V(W_d)=\{w\mid w\in\, \text{the sequence $W_d$}\}$. The vocabulary of a corpus is the union of the vocabularies of each document in that corpus; $V(C)=V(d_1)\cup V(d_2)\cup \dots\cup V(d_n)$.
\end{definition}

\begin{example}[Shapes Document Vocabulary]
  Using the same tokenization as before, the document from \autoref{ex:doc-shapes} has the vocabulary:
  \begin{align*}
    &V(\text{\emph{``$\triangle$ $\triangle$'', ``$\square$'', ``$\triangle$ $\triangle$'', ``$\pentagon$''}})\\
    &\quad=\{\text{\emph{``$\triangle$ $\triangle$''}},\, \text{\emph{``$\square$''}},\, \text{\emph{``$\pentagon$''}}\}
    \end{align*}
\end{example}
\vspace{6pt}

\begin{definition}[$\#$ Operator]
  The $\#$ symbol is used to denote the number of elements of the object it precedes. Before sequences it counts the number of elements in the sequence: $\#W_d=n$ (for $W_d=w_1,\dots,w_n$) \footnote{This is like the 'word count' feature many text editors provide}. Before sets it is a shorthand for the size of that set: $\#V(d)=|V(d)|=\text{the number of words in the vocabulary $V(d)$}$. Before a corpus it counts the number of documents in that corpus.
\end{definition}

\begin{example}[Shapes Document Sizes]
  The word count of the tokenized shapes document $$W_d=\text{\emph{``$\triangle$ $\triangle$'', ``$\square$'', ``$\triangle$ $\triangle$'', ``$\pentagon$''}}$$ is $\#W_d=4$ and the vocabulary size is $\#V(W_d)=3$.
\end{example}
\vspace{6pt}

\begin{example}[Plato Document Sizes]
  Word token counts and vocabulary sizes for a selection Plato documents from~\ref{ex:doc-plato}, tokenized using the simple method: \footnote{the code used to create these examples and others examples on the same corpus is available in the appendix} have 
  \vspace{6pt}
  \begin{table}[h]
    \centering
    \begin{tabular}{c r r r r r r}
      \toprule
      \multicolumn{1}{c}{Attribute$\quad$} &
      \multicolumn{6}{c}{Document ($d$)} \\

      \cmidrule(lr){2-7}
      &
      $\texttt{laws}$ &
      $\texttt{republic}$ &
      $\texttt{cratylus}$ &
      $\texttt{meno}$ &
      $\texttt{apology}$ &
      $\texttt{crito}$ \\
      \midrule
      $\#W_d$ & 140,406 & 118,283 & 23,883 & 12,716 & 11,392 & 5,332\\
      $\#V(W_d)$ & 8100 & 8,105 & 2,963 & 1,493 & 1,789 & 1,030\\
      \bottomrule
    \end{tabular}
    \caption{Number of Word Tokens \& Vocabulary Sizes of a selection of Documents in the Plato Corpus (\autoref{ex:corp-plato})}
  \end{table}
\end{example}
\vspace{6pt}

\begin{example}[Plato Corpus Size]
  The complete Plato corpus (\ref{ex:corp-plato}) consists of 25 documents and, each tokenized the same way as before, has the word count $\sum_{d\in C_P}\#W_d=670,936$ and the vocabulary size $\#V(C_P)=19,432$.
\end{example}
